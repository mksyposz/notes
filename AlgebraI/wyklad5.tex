\subsection{Zastosowanie wyznaczików macierzy}
\begin{tw} 
     $$
     (\ast) \left\{ 
        \begin{array}{l}
        a_{11} x_{1} + \dots a_{1n} x_{n} = b_1 \\
        \vdots \\
        a_{n1} x_{1} + \dots a_{nn} x_{n} = b_n 
        \end{array}
    \right.$$
    $$
    \text{inaczej}
    \begin{pmatrix} 
        a_{11} \ \dots \ a_{1n} \\
         \ \vdots \ \\ 
        a_{n1} \ \dots \ a_{nn} 
    \end{pmatrix}
    \begin{pmatrix}
        x_1 \\ 
        \vdots \\ 
        x_n
    \end{pmatrix}
    = 
    \begin{pmatrix}
        b_1 \\ 
        \vdots \\ 
        b_n
    \end{pmatrix} 
    \text{lub jeszcze inaczej } A\footnote{macierz A to macierz główna układu}X = b
    $$
    Układ $(\ast)$ ma jednoznacznie rozwiązanie $\Leftrightarrow$ $\det A \neq 0 $ i rozwiązanie ma postać 
    $x_i = \frac{\det A^{(i)}}{\det A}$, gdzie $A^{(i)} = (A_1,\dots,\overset{i}{b},\dots,A_n)$
\end{tw}

\begin{dd} 
    $$ A  \overset{ozn}{=} (A_1,\dots,A_n) $$ 
    $$(\ast) x_1A_1 + x_2A_2 + \dots + x_nA_n = b$$
    $$ \det A^(i) = det(A_1,\dots,\underset{i}{b},\dots,A_n) = \det(A_1,\dots,A_{i-1},\sum_{k=1}^n x_kA_k,A_{i+1},\dots,A_n) = $$
    $$ = \sum_{k=1}^n x_k \det(A_1,\dots,A_{i-1},A_k,A_{i+1},\dots,A_n) = x_i \det (A_1,\dots, A_{i-1},A_i,A_{i+1},\dots,A_n) = $$
    $$ = x_i = \det A$$
    $$ \det A^{(i)} = x_i \det A$$
Jeśli $det A \neq 0$ to $x_i = \frac{\det A^{(i)}}{\det A}$. \\
Jeśli $\det A = 0$, to $ker F \neq \{0\}, F: K^n \rightarrow K^n : F(X) = AX$

\begin{lem}
    Odwozrowanie $F: K^n \rightarrow K^n \text{t. że} F(X) = AX $ jest różnowartościowe/na/odwaracalne/ bijekcją ma trywialne jądro.$\Leftrightarrow \det A \neq 0$
\end{lem}
$ F^{-1} [b] = 
    \begin{cases} 
        \emptyset & \text{jeśli } b \notin \operatorname{Im} F\\ 
        \{v\} + \ker F & \text{jeśli } b \in \operatorname{Im} F,\text{ dla pewnego } v \in V
    \end{cases}$ \\
$ \dim ker F > 0 \Rightarrow \dim ImF < n \Rightarrow ImF \neq K^n $ \\
$b = F(v)$ dla pewnego $v \in V$ \\ 
$ \underset{w \in ker F}{F(v+w)} = F(v) + \underscript{F(w)}{\verteq}{0} = F(v) = b$ 
\begin{align}
    b &= F(v) \label{jeden}\\
    b &= F(v') \label{dwa} \\
    \shortintertext{Odejmując \eqref{jeden} od \eqref{dwa} otrzymujemy}
    0 &= F(v) - F(v') = F(v-v'), \text{ czyli } v-v' \in ker F \notag
\end{align}
    $v = v' + \underscript{(v - v')}{\vertin}{\ker F}$
\end{dd}

\begin{dd}
  (lematu)

  Jeśli $F$ jest odwracalna, to istnieje $G: K^n \rightarrow K^n$, $G(X)=BX$,
  takie że
  \[F \circ G = G \circ F = id\]
  \[m(F \circ G) = m(G \circ F) = I\]
  \[AB = BA = I\]
  czyli $B=A^{-1}$ ($A$ odwracalna).
  \[1 = \det (AB) \overset{\text{ćw}} = \det A \det B\]
  więc $\det A \neq 0$.

  W~drugą stronę ($\det A \neq 0 \Rightarrow A \text{ odwracalna} \land B \text{ odwracalna}$) to wniosek z~następnego faktu.
  \hfill \qed
\end{dd}
\begin{ft}
  $B = (b, j)$ jest macierzą odwrotną do $A$, wtedy i~tylko wtedy, gdy:
  \[b_{ji} = \frac1{\det A}(-1)^{i+j} +
  \begin{vmatrix}
  & & & \\ 
  & a_{ij} \makebox(-10,0){\rule[4mm]{0.4pt}{2.5\normalbaselineskip}} 
      \makebox(-10,0){\rule[2.5mm]{2.5\normalbaselineskip}{0.4pt}} 
  & \\ 
  & & &
  \end{vmatrix}
  \]
\end{ft}
\begin{prz}
  \[A = \begin{pmatrix}1&0&1\\0&2&1\\1&1&1\end{pmatrix}\qquad\det A = -1\]
  \[A^{-1}=\frac1{-1}\begin{pmatrix}1&+1&-2\\+1&0&-1\\-2&-1&2\end{pmatrix}^T
    =\begin{pmatrix}-1&-1&2\\-1&0&-1\\2&1&-2\end{pmatrix}\]
\end{prz}
\begin{dd}
  (faktu)

  \[AB=I\]
  \[A\cdot(B_1, \dots, B_n)=(e_1, \dots, e_n)\]
  \[AB_i=e_i \qquad e_i = \begin{pmatrix} 0 \\ \vdots \\ 1 \\vdots \\ 0\end{pmatrix} i\]
  układ Cramera.
  \[A = \begin{pmatrix}b_{1i}\\\vdots\\b_{ni}\end{pmatrix} i\]
    \[b_{ji}=\frac{\det(A_1,\dots,e_i,\dots,A_n}{\det A}=\frac1{\det A}i
    \begin{vmatrix}
      a_{11}&\dots&0&\dots&a_{1n}\\
      &&\vdots&&\\
      &&1&&\\
      &&\vdots&&\\
      a_{n1}&\dots&0&\dots&a_{nn}
    \end{vmatrix} = \frac1{\det A}1\cdot(-1)^{i+j}
    \begin{vmatrix}
  & & & \\ 
  & a_{ij} \makebox(-10,0){\rule[4mm]{0.4pt}{2.5\normalbaselineskip}} 
      \makebox(-10,0){\rule[2.5mm]{2.5\normalbaselineskip}{0.4pt}} 
  & \\ 
  & & &
  \end{vmatrix}
    \]
\end{dd}
\begin{uw}
  Operacje elementarne (wierszowe i~kolumnowe) nie wpływają na zerowanie się
  wyznacznika.
\end{uw}
\begin{ft}
  Dany jest \textit{jednorodny} układ równań liniowych ($a_{ij} \in K$).
  \[\text{(J)}\left\{\begin{array}{l}
      a_{11}x_1+\dots+a_{1n}x_n = 0
      \\\vdots\\
      a_{m1}x_1+\dots+a_{mn}x_n = 0
    \end{array}\right.\]
  Zbiór rozwiązań układu (J) jest podprzestrzenią $K^n$.
\end{ft}
\begin{dd}
  Oznaczmy $X = \begin{pmatrix}x_1\\\vdots\\x_n\end{pmatrix}$.
  \[F: K^n \rightarrow K^m, F(X)=AX\]
  zbiór rozwiązań to $\mathrm{ker}F < K^n$.
  \hfill \qed
\end{dd}
\begin{ft}
  Dany jest \textit{niejednorodny} układ równań liniowych
  ($a_{ij}, b_i \in K$):
  \[\text{(NJ)}\left\{\begin{array}{l}
      a_{11}x_1+\dots+a_{1n}x_n = b_1
      \\\vdots\\
      a_{m1}x_1+\dots+a_{mn}x_n = b_m
    \end{array}\right.\]
  Zbiór rozwiązań układu (NJ) jest albo $\emptyset$ albo ${v}+\mathrm{ker}F$,
  gdzie:
  \begin{itemize}
    \item $F: K^n \rightarrow K^m, F(X)=AX$
    \item $v$ - \textit{dowolne} rozwiązanie (NJ).
  \end{itemize}
\end{ft}
\begin{dd}
  \[F^{-1}[b]\]
  $1^\circ$ $b \notin \mathrm{Im} F$
  \[F^{-1}[b] = \emptyset\]
  $2^\circ$ $b \in \mathrm{Im} F$ 
  \begin{center}$b= F(v)$ dla pewnego (jakiegokolwiek) $v\in K^n$\end{center}

  $v'$ jest rozwiązaniem (NJ) $\Leftrightarrow F(v')=b$
  \[F(v+(v'-v)) = b\]
  \[F(v)+F(v'-v) = b\]
  \[b+\underscript{F(v'-v)}{\verteq}{0} = b\]
  \[v'-v \in \mathrm{ker}F\]
  \hfill \qed
\end{dd}
\begin{def}
  \textit{Warstwą} podprzestrzeni $W < V$ nazywamy każdy zbiór postaci
  $v + W \subseteq V$. Zbiór wszystkich warstw podprzestrzeni $W < V$
  oznaczamy $V/W$.
\end{def}  
\begin{ft}
  Jeśli na~przestrzeni liniowej $V$ wprowadzimy relację
  \[\sim_W: v\sim_Wv' \Leftrightarrow v-v'\in W\]
  (gdzie $W$ --- ustalona podprzestrzeń), to:
  \begin{enumerate}
    \item $\sim_W$ jest relacją równoważności
    \item $[v]_{\sim_W} = v+W$
  \end{enumerate}
  Dowód --- ćw.
\end{ft}
\begin{ft}
  $V/W$ z~działaniami:
  \begin{itemize}
    \item $(v+W)+(v'+W) = (v+v')+W$
    \item $\alpha(v+W) = (\alpha v) + W$
  \end{itemize}
  jest przestrzenią liniową (ilorazową).
\end{ft}
\begin{ft}
  $\dim(V/W) = \dim V - \dim W$
\end{ft}
\begin{dd}
  \[V \ni v \overset F\longmapsto v+W \in V/W\]
  \[\mathrm{ker}F = W\]
  twierdzenie o indeksie:
  \[\dim V = \underscript{\dim kerF}{\verteq}{\dim W} + \underscript{\dim\mathrm{Im}F}{\verteq}
  {\dim(V/W)}\]
  \hfill \qed
\end{dd}

\begin{prz}
    $$ \int: C([0,1]) \rightarrow C'([0,1]) \text{/ \{funkcje stałe\} }$$ 
\end{prz}

\begin{ft} 
    Układ równań (NJ): 
    \begin{enumerate}[{(}1{)}]
        \item jest sprzeczny $\Leftrightarrow Lin\{A_1,\dots,A_n\} \neq Lin\{A_1,\dots,A_n,b\}$ 
        \item jeśli jest niesprzeczny, to jego zbiór rozwiązań na wymiar $n - \dim Lin\{A_1,\dots,A_n\}$
    \end{enumerate}
\end{ft}

\begin{dd} \hfill
    \begin{enumerate}[{(}1{)}]
        \item $Lin\{A_1,\dots,A_n\} < Lin\{A_1,\dots,A_n,b\}$ \\
        Układ $AX = b$ ma rozwiązanie $\Leftrightarrow$ b jest kombinacją liniową $A_1,\dots,A_n \Leftrightarrow Lin\{A_1,\dots,A_n\} = \\ = Lin\{A_1,\dots,A_n\}$
        \item jeśli układ (NJ) niesprzeczny to zbiór rozwiązań to $v+ \ker F$ gdzie $\ker F$ to zbiór rozwiązań układu (J). 
        $$ F: K^n \rightarrow K^m,  F(X) = AX$$ 
        $$ \dim \ker F = \underscript{\dim K^n}{\verteq}{n} - \dim Im F = n - \dim Lin\{A_1,\dots,A_n\}$$ 
        \hfill \qed
    \end{enumerate}
\end{dd}

\begin{wn} ~\\
    Układ (NJ) jest sprzeczny \\
    \hspace*{20mm} $\Updownarrow$ \\
    $\dim Lin\{A_1,\dots,A_n\} < \dim Lin \{A_1,\dots,A_n\}$
\end{wn}

\begin{df}
    Rzędem (kolumnowym) macierzy $A \in M_{m \times n} (K) $ nazywamy wymiar podprzestrzeni $K^n$ rozpiętej przez kolumny macierzy A. \\ 
    Rzędem (wierszowym) nazywamy wymiar podprzestrzeni rozpiętej przez wiersze A. 
\end{df}

\begin{uw}
    Rząd wierszowy/kolumnowy = max. liczba lnz wierszy/kolumn.
\end{uw}

\begin{tw} 
    rząd wierszowym $A$ = rząd kolumnowy $A$.
\end{tw}

\begin{dd} \hfill
    \begin{enumerate}[{(}1{)}]
        \item rząd kolumnowy jest niezmienniczy na operacje kolumnowe
        \begin{align*}
            Lin\{A_1,\dots,A_n\} & = Lin\{A_1,\dots,A_j,\dots,A_i,\dots,A_n\} \\
            & = Lin \{A_1, \dots,A_i + \alpha A_j, \dots, A_n\} \\
            & = \underset{\alpha \neq 0}{Lin \{A_1, \dots, \alpha A_i, \dots, A_n \}}
        \end{align*}
        \item rząd kolumnowy jest niezmienniczy na operacje wierszowe
        $$
        \begin{pmatrix}
            1 \ 1 \\ 
            1 \ 1 \\ 
            1 \ 0 
        \end{pmatrix}
            \rightarrow
        \begin{pmatrix}
            1 \ 1 \\ 
            0 \ 0 \\ 
            1 \ 0
        \end{pmatrix}
        $$
        operacja wierszowa: $$ A \rightarrow EA $$ E - macierz elementarna
        $$A_1, \dots A_n \rightarrow EA_1,\dots,EA_n$$ 
        $$ F_E : K^n \rightarrow K^n $$
        $$ X \mapsto EX $$
        $F_E$ jest odwracalne ( bo $\det E \neq 0$ ) tzn. jest bijekcją. \\
        W szczególności $\dim W = \dim F_E (W)$ dla dowolnego $ W < K^n$, \\ 
        czyli $\dim Lin \{A_1,\dots,A_n\} = \dim Lin \{EA_1,\dots,EA_n\}$.
        \item rząd wierszowy również jest niezmienniczy na operacje wierszowe i kolumnowe (bo rząd wierszowy $A$ = rząd kolumnowy $A^T$).
        \item dowolną (prostokątną) macierz $A$ możemy przy pomocy operacji wierszowych i kolumnowych sprowadzić do postaci $$
        \begin{pmatrix}
        1 & & & & & & \\ 
        & \ddots & & & & &  \\
        & & 1 & & & & \\
        & & & 0 & & & \\ 
        & & & & \ddots & & \\ 
        & & & & & 0 \dots \\
        \end{pmatrix} 
        $$
        a tu rząd kolumnowy = rząd wierszowy. \qed
    \end{enumerate}
\end{dd} 

\begin{ozn}
    rank A= rząd( wierszowy/kolumnowy) macierzy A
\end{ozn} 

\begin{tw} (Kronecker-Capelli) \\ 
    Układ (NJ) ma rozwiązanie $\Leftrightarrow \overset{\text{macierz główna układu}}{rank A} = \overset{\text{macierz rozszerzona układu}}{rank(A|b)}$
\end{tw}

\begin{dd}
    Już był. \qed
\end{dd}

\begin{tw}
    Jeśli układ (NJ) ma rozwiązanie to wymiar zboru rozwiązan wynosi $n - rank A$. 
\end{tw}
\underline{W szczególności:} układ m lnz równań z n niewiadomymi ma zbiór rozwiązań wymiaru n-m.

\begin{ft}
    $v_1,\dots,v_n \in K ^n $ lnz $ \Leftrightarrow \det(v_1,\dots,v_n) \neq 0$.
\end{ft}
\begin{dd} 
    $$ v_1, \ldots, v_n \text{ są lz}$$
    $$\Updownarrow$$ 
    $$ \operatorname{rank} (v_1, \dots , v_n) < n $$  
   $$ \Updownarrow $$  
   $$ \det (v_1,\dots v_n) = 0 $$ \hfill \qed 
\end{dd} 

\begin{tw} Rozwiązanie ogólne układu $AX = b$ jest postaci $v + W$, gdzie \\
    W - przestrzeń rozwiązań $AX = 0 \ (\ker A)$\\ 
    v - jakiekolwiek rozwiązanie AX = b (rozwiązanie szczególne)
\end{tw} 
\subsection{Metoda eliminacji Gaussa} 
    $(A |b)$ macierz rozszerzona układu    
    \begin{itemize} 
        \item[Krok 1]   Sprowadzenie $(A |b)$ do postaci schodkowej, przy pomocy operacji \underline{wierszowych}. 
        $c_i \neq 0$ - wyrazy wiodące. Zmienne odpowiadające kolumnom z wyrazami wiodącymi nazywamy zmiennymi
        zależnymi. Pozostałe zmienne to zmienne niezależne, alb zmienne wolne. 
        Liczba zmiennych wolnych = $n - rank A$. 
        \item[Krok 2] Wyliczyć zmienne zależne przy pomocy zmiennych wolnych i zapisać rozwiązanie ogólne.
    \end{itemize} 
